%=======================================
%TODO Frame 1
\begin{frame}[b]

\deuxcolonnesbottom{
\begin{center}
\includegraphics[width=1.0\textwidth]{ghostbusters-full.png}
\end{center}

\vspace{0.7cm}

\mysection{Introduction}

\myindent \href{https://en.wikipedia.org/wiki/Ghostbusters_(role-playing_game)}{Ghostbusters} est un jeu de rôles humoristique américain, conçu par \href{https://en.wikipedia.org/wiki/Sandy_Petersen}{Sandy Petersen}, \href{https://en.wikipedia.org/wiki/Lynn_Willis}{Lynn Willis} (les créateurs du jeu de rôles \href{https://en.wikipedia.org/wiki/Call_of_Cthulhu_(role-playing_game)}{Call of Cthulhu}), et \href{https://en.wikipedia.org/wiki/Greg_Stafford}{Greg Stafford} (le créateur de \href{https://en.wikipedia.org/wiki/RuneQuest}{Runequest} et de \href{https://en.wikipedia.org/wiki/Pendragon_(role-playing_game)}{Pendragon}), et publié en 1986 par \href{https://en.wikipedia.org/wiki/West_End_Games}{West End Games}.

\myindent Le système de jeu est très compact et très simple et vise des aventures amusantes dans l'univers des films. Il servira de fondation au \href{https://en.wikipedia.org/wiki/D6_System}{Système D6} dont \href{https://en.wikipedia.org/wiki/Star_Wars:_The_Roleplaying_Game}{Star Wars} est l'exemple le plus connu, et inspirera des dizaines de mini-jeux de rôles comme \href{https://rouboudou.itch.io/risus}{Risus}.

\myindent Ce document reprend les règles originales de 1986.

\vspace{0.2cm}

\begin{tabular}{p{3cm}p{5cm}}
Concepteur               & \copyright\ S. Petersen, L. Willis, G. Stafford \\
Version        &  Version originale 1986 \\
Traduction et adaptation & \textcopyleft\ O. Rey 2022 \\
Version                  & \myversion \\
Référence         & \myreference \\ 
\end{tabular}

\vspace{0.4cm}

%=================================================SECTION
\mysection{Création du personnage}

\mysubsection{Traits}

\myindent Les Ghostbusters possèdent quatre \textbf{Traits} :
\begin{myitemize}
\item Sa \textbf{Cervelle} (Brains),
\item Ses \textbf{Muscles} (Muscles),
\item Ses \textbf{Mouvements} (Moves),
\item Sa \textbf{Coolitude} (Cool).
\end{myitemize}

\myindent On assigne à chaque Trait une valeur numérique. Plus cette valeur est haute, meilleure est la performance du PJ qui l'utilise. Chaque joueur possède 12 points à répartir sur les 4 traits, sachant que la valeur d'un trait ne peut être ni inférieure à 1, ni supérieure à 5.

\myindent \textit{Note} : certains Ghostbusters ont plus de 5 points dans certains Traits, mais c'est le fruit de l'expérience (voir plus loin).

\myindent Lors d'une action dont l'issue n'est pas connue d'avance, le Ghostmaster assignera un \textbf{Facteur de Difficulté} (FD) à l'action. Le joueur devra lancer un nombre de \textbf{dés à 6 faces} correspondant au Trait associé, et son score devra être supérieur ou égal au FD pour que l'action réussisse. Un de ces dés doit être un \textbf{Dé Fantôme} (DF), un dé à 6 faces de couleur différente ayant un fantôme à la place du 6 (ou géré comme tel).

\mysubsection{Talents}

\myindent Les Ghostbusters ont un \textbf{Talent} par Trait. Quand le PJ tente une action pour laquelle il possède un Talent, le joueur lancera 3D de plus que son Trait.

\myindent Chaque joueur doit choisir un Talent par Trait dans les listes ci-dessous. Il est possible de choisir un Talent non listé si le MJ l'accepte (dans la plupart des scénarios de Ghosbusters RPG, les PNJ ont des Talents ne figurant pas dans ces listes).
}{
\begin{center}
\begin{tabular}{ccc}
&\textbf{Cervelle}&\\
Anthropologie & Déduction & Occultisme \\ 
Archéologie & Deviner & Parapsychologie \\
Astronomie & Électronique &  Physique \\ 
Bibliothèque scientifique & Faits sportifs & Psychanalyse \\ 
Biologie & Géologie & Réparation électrique \\
Botanique & Histoire & Réparation mécanique \\
Bureaucratie & Journalisme & Romans à l'eau de rose \\
Chimie & Linguistique & Zoologie \\
Coiffure à la mode & Mathématiques & \\
Comptabilité & Médecine & \\
\end{tabular}
\end{center}

\begin{center}
\begin{tabular}{cc>{\centering\arraybackslash}p{2.5cm}}
& \textbf{Muscles} &  \\
Bagarre & Escalader & Lutte \\
Casser des choses & Éventrer & Nager \\
Renverser à coups de pieds & Gloutonnerie & Sauter \\
Courir & Intimider & Soulever \\
\end{tabular}
\end{center}

\begin{center}
\begin{tabular}{ccc}
& \textbf{Mouvements} & \\
Attirer l'attention & Faire de la musique & Se déguiser \\
Attraper & Fouiner & Se pavaner \\
Breakdance & Furtivité & Séduire \\
Conduire un véhicule & Jeter & Tirer avec une arme \\
Écouter & Pickpocket & Tours de passe-passe \\
Équilibre & Ragots & Voir \\
Esquiver & Se cacher &  \\
\end{tabular}
\end{center}

\begin{center}
\begin{tabular}{>{\centering\arraybackslash}p{2.5cm}cc}
& \textbf{Coolitude} & \\
Bluffer & Élever des enfants & Jouer à la Bourse \\
Charmer & Emprunter & Marchander \\
Confusionner & Intimider & Orateur \\
Convaincre & Jouer au Poker & Raconter des bobards \\
\end{tabular}
\end{center}

\mysubsection{Bons Points}

\myindent Tous les Ghostbusters commencent le jeu avec 20 \textbf{Bons Points} (BP). Des BP seront gagnés quand ils auront fini les missions ou auront achevé leurs objectifs.

\myindent Quand les Ghostbusters vont mal, ils perdent des BP et prennent des amendes pour stationnement interdit, des appels injurieux de leurs créditeurs ou de longs séjours à l'hôpital.

\myindent Les Bons Points sont plus qu'une mesure de l'état de santé des Ghostbusters, ce sont des moyens pour eux d'influer sur l'histoire, de les voir tenter d'incroyables prouesses ou de les sortir de terribles ennuis.

\begin{center}
\begin{tabular}{cc}
\textbf{Action} & \textbf{Coût en BP}\\
Augmenter ses chances de succès\textsuperscript{1} & 1 BP = +1D au jet \\
Réduire le temps de séjour à l'hôpital & 1 BP = $-1$ semaine \\
Compenser une action très stupide & Un certain nombre de BP \\
Éviter de mourir & Un certain nombre de BP \\
\end{tabular}
\end{center}

{\small \textit{\textsuperscript{1} La décision de dépenser des Bons Points doit être faite avant le jet et ne peut pas modifier un jet déjà réalisé.}}

\mysubsection{Liens entre Bons Points et Traits}

\myindent Les Traits sont liés avec les BP de la manière suivante (avec accord du Ghostmaster) :

\begin{center}
\begin{tabular}{ccc}
\textbf{Contexte} & \textbf{Action} & \textbf{Impact en BP}\\
Expérience & Faire +1 sur un Trait & $-30$ BP \\
Moment critique\textsuperscript{2} & Faire $-1$ sur un Trait & +20 BP \\
\end{tabular}
\end{center}

{\small \textit{\textsuperscript{2} Par exemple, au lieu de mourir, un Ghostbuster n'ayant plus de BP peut en retrouver en perdant définitivement 1 point sur un de ses Traits.}}

\myindent Quand les BP sont utilisés par les joueurs pour sortir le Ghostbuster d'une mauvaise situation, ces derniers doivent décrire la situation. Si la description est amusante, le Ghostmaster pourra gratifier le Ghostbuster d'un bonus de +1 ou +2 BP.

\mysubsection{Expérience}

\myindent Le gain de BP après une mission est évalué comme suit :

\begin{center}\begin{tabular}{>{\centering\arraybackslash}p{3cm}>{\centering\arraybackslash}p{5cm}}
\textbf{Situation} & \textbf{Gain en BP}\\
Mission échouée ou fantôme non capturé & La moitié des BP perdus pendant la mission \\
Mission réussie & La totalité des BP récupérés, voire un léger bonus \\
Mission accomplie avec panache et fun & Tous les BP sont récupérés + la moitié des BP perdus pendant la mission \\
\end{tabular}
\end{center}


}
\end{frame}

%=======================================
%=======================================
%TODO Frame 2
\begin{frame}[b]

\deuxcolonnesbottom{%col1

%\begin{minipage}[c][0.95\textheight][c]{\linewidth}

\mysubsection{Objectif Personnel}

\myindent Le Ghostbuster poursuit un \textbf{Objectif Personnel}. L'atteindre lui accordera de nouveaux BP. La liste est donnée ci-dessous, sachant que le Ghostmaster peut valider un objectif n'étant pas dans la liste.

\myindent \textbf{Sexe} -- Le Ghostbuster est habitué à des relations sans lendemain. Si le Ghostbuster réussit à avoir un rendez-vous amoureux qui se passe bien durant une aventure : +1DF en BP. Si le Fantôme est tiré, le Ghostbuster fait une grosse bêtise pendant le rendez-vous et ne gagne pas de BP (mais en gagnera un de plus au prochain jet réussi).

\myindent \textbf{Richesse} -- Le Ghostbuster est attiré par les richesses. Le Ghostmaster peut récompenser le Ghostbuster qui gagne de l'argent ou qui gagne à la bourse. Bien entendu, en cas de retournement de fortune, le Ghostmaster pourra pénaliser le Ghostbuster de quelques BP.

\myindent \textbf{Célébrité} -- Le rêve du Ghostbuster est d'être un héros des médias, d'être reçu par le Président, etc. Ce dernier gagne 1DF de BP quand il passe dans les médias locaux et 2D quand il passe dans les médias nationaux (dont le DF). S'il fait la une d'un journal international comme \textit{Time}, il peut gagner jusqu'à 6DF ou plus (dont le DF) ! En cas de tirage du Fantôme, le Ghostbuster est ridiculisé et il ne gagne aucun BP.

\myindent \textbf{Science sans conscience} -- Le Ghostbuster pense que la science passe avant tout, sans préoccupation pour l'impact sur les gens alentour. Une découverte en fantômologie fera gagner 1DF en BP au Ghostbuster, et une découverte majeure pourra lui faire gagner plus (à la discrétion du Ghostmaster). Si le Fantôme est tiré, aucun gain de BP n'est réalisé mais un point de plus sera gagné au prochain jet réussi.

\myindent \textbf{Servir l'Humanité} -- Le Ghostbuster est intéressé par faire le bien, sauver le monde, etc. A la fin de chaque aventure, le Ghostbuster reçoit 1DF de BP par ennemi de l'humanité vaincu. Si l'ennemi était particulièrement coriace, il est possible que Ghostmaster accorde plus. Une victoire sur une agence gouvernementale peut aussi apporter un bonus de BP.

\mysubsection{Personnalité et photo}

\myindent Le joueur doit déterminer la personnalité de son Ghostbuster, potentiellement ajouter une photo, remplir sa ficher de personnage et le tour est joué.

%=================================================SECTION
\mysection{Les règles}

\mysubsection{Faire des choses}

\myindent Les tâches pour lesquelles l'issue n'est pas évidente se voient attribuer un Facteur de Difficulté par le Ghostmaster suivant la table ci-dessous :

\begin{center}
\begin{tabular}{>{\centering\arraybackslash}p{3cm}>{\centering\arraybackslash}p{3cm}}
\textbf{Type de tâche} & \textbf{FD}\\
Tâche facile & 5 ou plus \\
Tâche normale & 10 ou plus \\
Tâche difficile & 20 ou plus \\
Tâche impossible & 30 ou plus \\
\end{tabular}
\end{center}

\mysubsection{Influence du Dé Fantôme}

\begin{center}
\begin{tabular}{p{3.5cm}p{4.5cm}}
\textbf{Type de tirage} & \textbf{Conséquence}\\
Pas de Fantôme & Ajouter les chiffres et comparer au FD \\
Fantôme et somme des autres dés supérieure ou égale au FD & Succès accompagné d'un petit quelque chose d'ennuyeux \\
Fantôme et somme des autres dés inférieure au FD & Échec et quelque chose de mauvais arrive \\
\end{tabular}
\end{center}

\myindent Le Dé Fantôme marche de manière inversée pour les revenants : il garde sa valeur 0 mais provoque quelque chose de bon pour eux.

\mysubsection{Duels}

\myindent Le duel peut se dérouler sur tous les niveaux : mots, poings, cervelle, habileté, argent, ou quoique ce soit d'autre.

%\end{minipage}
}
{
\myindent Le Ghostmaster décide quel Trait ou Talent est applicable pour chacun des protagonistes. Chaque duelliste lance ses dés (incluant le DF) et le plus haut score gagne. Si les protagonistes sont ex-æquo, selon les cas, le Ghostmaster laissera en l'état ou fera rejouer. Il traitera aussi le cas de tirage de Fantômes.

\mysubsection{Mouvement}

\myindent Plus le score de Mouvements est élevé et plus le Ghostbuster est rapide. La gestion des mouvements est laissée à l'appréciation du Ghostmaster.

\mysubsection{Séquence de jeu}

\myindent La séquence de jeu est la suivante :

\begin{center}
\begin{tabular}{c p{7.5cm}}
\textbf{\#} & \textbf{Conséquence}\\
1 & Le Ghostmaster annonce ce que vont faire les PNJ (personnages ou fantômes) \\
2 & Le Ghostmaster demande à chaque joueur en commençant par sa droite ce que les Ghostbusters comptent faire \\
3 & Le Ghostmaster décide qui va faire quoi à qui et dans quel ordre et quels dés doivent être lancés \\
\end{tabular}
\end{center}

\myindent Durant les séquences d'action, chaque PJ peut faire un déplacement et une action.

\mysubsection{Combat}

\myindent Il y a deux types de combats : les \textbf{combats de mêlée} et les \textbf{combats à distance}. Le Talent \textit{Bagarre} est le plus adapté aux combats de mêlée, ou le Trait \textit{Muscles}. Si un adversaire ou un PJ utilise un gourdin, alors ce dernier aura un avantage plus ou moins important (voir table ci-dessous).

\begin{center}
\begin{tabular}{c p{6.5cm}}
\textbf{Bonus} & \textbf{Armes}\\
+1D & Poing américain, matraque, ongles longs \\
+2D & Cran d'arrêt, fouet, poêle à frire \\
+3D & Gourdin, chaise, épée \\
+4D & Hache de bataille, tronçonneuse, perceuse électrique \\
\end{tabular}
\end{center}

\myindent Les combats à distance utilisent des armées ayant une certaine portée qui pilote la difficulté du jet à effectuer.

\begin{center}
\begin{tabular}{l p{5.5cm} c}
\textbf{Distance} & \textbf{En mètres} & \textbf{FD} \\
Bout-portant & Moins de 3m & 5 \\
Normale & Entre les deux & 10 \\ 
Longue & 15m ou plus pour un pistolet, fusil à canon scié, or canon à proton (\textit{proton-pack}), 150m pour un fusil &  20 \\
\end{tabular} \\
\end{center}

\mysubsection{Mort, dommages, hôpital et maladie}

\myindent Les Ghostbusters ne meurent généralement jamais, les fantômes non plus d'ailleurs (ils le sont déjà). Par contre, ils sont blessés, le matériel est endommagé ou détruit et les fantômes sont engloutis par le piège à fantômes. Pour autant, en cas d'abus d'un joueur, le Ghostmaster pourra faire mourir le PJ qui, après des funérailles touchantes, pourra hanter ses anciens amis.

\begin{center}
\begin{tabular}{p{2cm} p{6cm}}
\textbf{Dommages} & \textbf{Perte en BP}\\
Touché par une hache de guerre & De 1 à 10 (membre sectionné mais recollable en hôpital) suivant la gravité \\
Chute & 1 BP par 5 mètres avec un maximum de 5 BP (quelque soit la hauteur) \\
Feu, radioactivité & 1 BP par brûlure avec un maximum de 20. La radioactivité a un effet retard (le MJ peut la gérer sans que les joueurs ne soient au courant jusqu'à apparition des symptômes) \\
Noyade, asphyxie & 1 BP par minute sans respirer jusqu'à un maximum de 10 \\
Poison & 1 BP si le poison fait juste mal ventre, jusqu'à 10 ou 15 BP pour les poisons les plus violents \\
\end{tabular}
\end{center}

\mysubsection{Équipement}

\myindent Les Ghostbusters ne peuvent porter que 3 objets sans pénalité (au maximum un nombre d'équipements égal à leurs \textit{Muscles} mais sans possibilité d'action autre que bouger et grommeler).

%\begin{minipage}[c][0.95\textheight][c]{\linewidth}


%\end{minipage}
}
\end{frame}

%=======================================
%=======================================
%TODO Frame 3
\begin{frame}[b]

\deuxcolonnesbottom{%col1

\begin{wrapfigure}{l}{5cm}
%\centering
\includegraphics[width=0.5\textwidth]{./images/ECTO-1.png}
\end{wrapfigure}

\myindent Les Ghostbusters possèdent une voiture toute équipée (ECTO-1) qui normalement propose 3 places confortables, pour monter à 6 dans les moments critiques.

\myindent Par contre, cet engin est tout sauf discret.

\begin{center}
\begin{tabular}{>{\raggedright\arraybackslash}p{1.8cm} p{6cm}}
\textbf{Équipement} & \textbf{Caractéristiques}\\
Kit d'escalade & +3D en \textit{Muscles} ou \textit{Escalade} \\
& Fonctionne sur les montagnes, les buildings, les ascenseurs. Comporte des pitons, des marteaux, des cordes, des harnais et même un short bavarois\\
Analyseur d'auras & Un écran et un casque. Révèle les sentiments et les mensonges de celui qui porte le casque. Révèle l'archétype du personnage ; s'il est possédé, l'archétype de ce qui le contrôle.\\
Kit de plage & Un ballon et un filet de volleyball ; des lunettes de soleil, de la crême solaire, un parasol, des grandes serviettes, un frisbee, une radio, etc. \\
Un mégaphone & Pour se faire entendre des gens et des fantômes \\
Une connexion haut-débit Internet satellite & Dans la VO, il s'agit d'un téléphone de voiture, signe d'un certain niveau social (en 1986). \\
Des lunettes d'Ecto-Vision & Permettent de voir dans le noir. Éliminent la vision latérale. Font paraître un peu idiot.\\
Compteur \newline Geiger & Pour les endroits radioactifs ou les accélérateurs de particules pirates \\
Piège à fantôme & Une petite boîte reliée à une pédale par un fil d'environ 3 mètres. Quand on appuie sur la pédale, le piège s'ouvre et produit un cône de force psycho-cinétique attirant le fantôme à l'intérieur. Mais comme le fantôme peut bouger et s'éloigner, il est bon de le retenir avec un canon à protons. Quand on lève le pied de la pédale, le piège se referme. \\
\end{tabular}
\end{center}

\begin{center}
\includegraphics[width=0.4\textwidth]{./images/aura-analyseur.png} \hspace{0.5cm}  \includegraphics[width=0.28\textwidth]{./images/ghost-trap.png}\textsl{}
\end{center}



}{%col2

}
\end{frame}
